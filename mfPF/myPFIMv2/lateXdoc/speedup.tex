\section{Some possible speedups (without using C-code) and numerical improvements}
\subsection{Jacobian vs Hessian}
The main speedup between PFIM and myPFIMv2 is that for calculating the numerical sensitivities (model.sensNum) I use the function \verb!jacobian()! from the package {\it numDeriv}.
This derives all sensitivities within one function call!
In the original PFIM the function \verb!fdHess()$gradient! from the package {\it nlme} which first derives the Hessian and then the gradient is approximated. 
Additionally this is done per time-point so for many sampling times this is very time consuming.


\subsection{Parallel solving of ODE and sensitivities}
As mentioned before: with some mathematics one can get a 100-fold speedup regarding solution of the sensitivities, but it needs some patience as well.
See examples model.defaultDEjacY and model.sensJac for details.

\subsection{invvar via eigendecomposition}
As described above one should avoid deriving inverse matrices at all.
I currently still derive the inverse, but as a pseudo-inverse using eigendecomposition.
This provides much more robust results (and no crash of the program after 20 minutes of calculation\dots)

\subsection{Use of symmetries when deriving PFIM}
Almost all matrices used in the model.fish.bloc function are symmetrical.
This should be used to speed up the computation (only minor improvements as matrices are small) and also to increase numerical accuracy (avoid divisions, etc.)

The main change here was that as mentioned before the trace of a matrix multiplication of symmetric matrices can be derived by the sum of their element-wise product, i.e.,
\verb!sum(diag(A %*% B)) == sum(A * B)!
